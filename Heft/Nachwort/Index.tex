\section{Das war's}
Hoffentlich kannst Du jetzt kochen! Was ließt du das hier überhaupt noch? Wenn du gerade etwas gekocht hast, solltest du es lieber schnellstens essen, denn:

\includegraphics[width=\textwidth]{_assets/Stoll/Kaltes-Essen.png}

\section{Verbessere mich!}
Diese Sammlung an wunderbaren Rezepten findest du im Internet unter \href{https://github.com/pr0nopoly/rezepte}{github.com/pr0nopoly/rezepte}. Falls Dir in Deinem Keller mal wieder langweilig ist, kannst du das pr0gramm nach ein paar weiteren tollen Rezepten durchsuchen und diese hier hinzufügen.

Ich war zum Ende hin ziemlich unter Zeitdruck, deshalb konnte ich einige Stellen nicht so machenm wie ich sie machen wollte. Da sollte man sich dann später noch mal drum kümmern.

\subsection{Technische Details}
Diese Rezeptsammlung ist ein Haufen \LaTeX-Dateien. Jedes Rezept ist in einer separaten Datei. Anschließend wird alles mit TexMaker bzw. TeXnicCenter zu einer PDF kompiliert. Mit "pandoc"{} wird außerdem eine HTML-Version für die GitHub-Pages auf
\begin{center}
	\href{https://pr0nopoly.github.io/rezepte}{pr0nopoly.github.io/rezepte}
\end{center}
erstellt.
Leider habe ich keinen wirklichen Plan von \LaTeX. Dieses Fontrendering ist nur so toll (und die Tatsache, dass das \LaTeX-Logo ein eigenes Command hat). \LaTeX \LaTeX \LaTeX \LaTeX \LaTeX \LaTeX \LaTeX \LaTeX. Ich hoffe, dass Jemand irgendwann mal mein hässliches Markup fixen wird. :/
