\prorezeptheader{
	Pizza-Suppe
}{
	vverner
}{
	465487
}{
	\diffneuschwuchtel
}

\hspace{10mm}

\prorezeptmenge{5}{2}

\hspace{10mm}

\prorezeptzutaten{
	\item 1kg Hackfleisch
	\item 3 Paprikaschoten
	\item 5 Zwiebeln
	\item 2 Dosen Champignons
	\item 2 Dosen Tomaten
	\item 2 Becher Sahne
	\item 1 Becher Schmand
	\item 1 Packung Schmelzkäse
	\item 1 Packung Schmelzkäse ''Kräuter''
	\item Salz/Pfeffer
	\item Pizzagewürzmischung
	\item Etwas Öl zum Braten
}
\prorezeptequipment{
	\item Topf
	\item Rührlöffel
	\item Messer
	\item Schneidebrett

	\propostqrinline{465487}
}

\hspace{10mm}

\promisctext{Ja, für das Rezept bleibt es leider nicht aus, den Keller zu verlassen. Sorry dafür! Dauert Alles in Allem ca. eine Stunde, aber lohnt!}

\prorezeptzubereitung{
	\item Zwiebeln schälen, klein schneiden und im Topf mit einem guten Schuss Öl anbraten, bis gewünschte Bräunung erreicht. Bei mir gibt's am Herd 12 Stufen, habe ihn auf die 10. gestellt.
	\item Champignons (ich habe hier 1x aus der Dose und 1x frische genommen) waschen und schneiden. Auch die Paprika waschen und schneiden.
	\item Jetzt kommt das Hack in den Topf (\proimportant{Nachölen!}) Einfach die Zwiebeln drauf, ein bisschen klein machen und anbraten. Sollte am Ende mit den Zwiebeln vermengt sein. Salz und Pfeffer dran und Pizzagewürzmischung auch.
	\item Nun noch die Paprika, Tomaten und Pilze dazu und alles gut verrühren.
	\item Als Letztes die Sahne, den Schmand (Crème Fraîche geht auch) und den Schmelzkäse rein. Nochmal verrühren.
	\item Ca. \produration{30 Minuten} auf halber Hitze kochen lassen, zwischendurch noch umrühren.
}

\promisctext{Voilà! Fertig ist die Flüssigpizza! Als Teigersatz einfach Brot/Pr0t dazu. Auch für Kellerpartys bestens geeignet.

\noindent Guten Appetit!}

\subsection{OC}
\proocsmallside{465487-0}{465487-1}
