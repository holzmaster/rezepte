\prorezeptheader{
	Feuerzangenbowle
}{
	PetersPan
}{
	463902
}{
	\difflegende
}

\hspace{10mm}

\promisctext{Eine Rezeptempfehlung für die kalte, einsame Zeit im Keller.}

\hspace{10mm}

\prorezeptzutaten{
	\item 1 Zuckerhut
	\item 2 Tetrapacks Rotwein
	\item 1 Flasche 54\%igem Rum (ca. Proof 100)
	\item 1-2 Orangen (nicht gespritzt/gewachst)
	\item 1 Zitrone (nicht gespritzt/gewachst)
	\item 3-4 Zimtstangen
	\item 1 handvoll Sternanis
	\item ein paar Nelken
}
\prorezeptequipment{
	\item 1 großer Topf
	\item 1 Herdplatte oder Campingkocher
	\item 1 Feuerzange (alte Reibe)
	\item 1 Kelle aus Metall
	\item 1 große Schüssel \proimportant{(kein Plastik!)}
	\item Feuerzeug oder Streichhölzer

	\propostqrinline{463902}
}

\hspace{10mm}

\promisctext{So, wenn Ihr mal wieder einsam im Keller hockt, Ihr nicht als das pr0gramm und Pr0ns habt, hilft eigentlich nur eins: \proimportant{Alkohol!} (Wenn Ihr es schafft, noch ein paar andere nette Leute außer Eure Sexpuppe in den Keller zu locken wird's noch lustiger).

\noindent Also hurtig alle Sachen besorgt und los geht's. \proimportant{Wie, Ihr habt keine Feuerzange?} Dann nehmt einfach die alte Reibe, die Ihr über den Topf legen könnt. Wenn Ihr die auch nicht besitzt, weil das Essen ja bekanntlich vom Muselmann oder der Mama gebracht wird, geht Ihr mal zu einer der vielen alten Ommas, die in der Gegend wohnen, und fragt, ob Ihr Euch eine Reibe leihen könnt. Keine Sorge, die alte, senile Schabracke hat eh Alzheimer und wird vergessen, dass Ihr ihr die Reibe nicht wiederbringt.}

\prorezeptzubereitung{
	\item Orangen und Zitrone waschen und in Scheiben schneiden.
	\item Wein mit Orangen, Zitrone und Gewürzen im Topf erhitzen (\proimportant{bloß nicht kochen lassen!}). Bei dem Anis und den Nelken einfach eine handvoll von beidem nehmen und aus 3m Entfernung ich Richtung Topf schmeißen. Was reinfällt ist die perfekte Menge.
	\item Zuckerhut auf die Feuerzange über dem Topf legen, mit Rum übergießen und anzünden. Wenn Ihr hier verkackt habt und der Rum nicht stark genug ist, brennt er natürlich nicht.
	\item Den restlichen Rum in die große Schüssel gießen, neben den Topf stellen und mit der Kelle immer weiter über den Zuckerhut gießen. \proimportant{Achtung:} Sowohl Kelle als auch der Rum in der Schüssel können mal Feuer fangen. Einfach auspusten oder den Keller abfackeln lassen. Dies solange machen, bis sich der Zuckerhut vollständig aufgelöst hat.
	\item Je nach Geschmack noch einen Zuckerhut mit Rum in die Bowle brennen.
}

\promisctext{Prost!}

\subsection{OC}
\prooc{463902-0}
