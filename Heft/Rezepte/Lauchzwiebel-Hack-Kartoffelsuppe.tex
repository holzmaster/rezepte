\prorezeptheader{
	Lauchzwiebel-Hack-Kartoffelsuppe
}{
	wursttrommler
}{
	474297
}{
	\diffaltschwuchtel
}

\prorezeptmenge{6-8}{1-2}

\hspace{10mm}

\prorezeptzutaten{
	\item 600g Hackfleisch (gemischt oder Rind)
	\item 1 große Gemüsezwiebel
	\item 2 Bund Lauchzwiebeln
	\item 1-2 Standen Lauch
	\item 4 große Kartoffel, festkochend
	\item 400g Schmelzkäse
	\item 1 Becker Crème Fraîche oder saure Sahne
	\item 1.2-1.5l Wasser
	\item 3-4 Würfel oder Teelöffel Gemüsebrühe
	\item Salz/Pfeffer
	\item Muskatnuss, gerieben
}
\prorezeptequipment{
	\item Abgewaschenes Geschirr
	\item Messer
	\item Schüssel
	\item Herd
	\item Pfanne
	\item Kochtopf
	\item Hässlicher Oma-Teller

	\propostqrinline{474297}
}

\hspace{10mm}

\promisctext{Zubereitungszeit je nach Behinderungsgrad 20 bis 40 Minuten.}

\prorezeptzubereitung{
	\item Als Erstes solltet Ihr das ranzige Geschirr in Eurer Spüle abwaschen. Den Platz und das Zeug braucht Ihr noch.
	\item Untere (weiße) zwei Drittel der Lauchzwiebeln kleinschneiden, ''normale'' Zwiebel schälen und kleinwürfeln.
	\item Zwiebelgeschnittenes abseits in einer Schüssel aufbewahren.

	Obere (grüne) zwei Driffel der Lauchzwiebeln und das komplette Lauch (Porree) längs aufschneiden und waschen.Lauchgewächse haben mehr Sand in den Löchern also so mancher Admin hier. Wenn alles sauber ist, kleinschneiden.
	\item Zwiebelzeugs anbraten, regelmäßig umrühren, damit nichts anbrennt. Bei gewünschten Gargrad Hack hinzufügen und kleinmachen.
	\item Das Wasser zum Kochen aufsetzen.

	Sobald das Hack ein bisschen Farbe gewonnen hat, Lauchgestrüpp hinzufügen und kurz angehen lassen. Das heiße Wasser und die Gemüsebrühe hinzufügen.
	\item Während das Ganze so vor sich hinköchelt, die Kartoffeln schälen und in kleine Würfel schneiden. Ich habe rote Kartoffeln genommen, weil is so. Die dann mit reinschmeißen.
	\item Das Ganze darf jetzt \produration{20 Minuten} vor sich hinköcheln.
	\item Danach den Schmelzkäse (alternativ Sahne), die Crème Fraîche / saure Sahne hinzufügen und kräftig aufkochen lassen. Dann abschmecken, mit Salz, Pfeffer und Muskat würzen. Bei der Mange habe ich fast eine halbe Muskatnuss hineingerieben.

	Baut keinen Unsinn, Kinder. Muskatnuss kann ganz, ganz üble Nebenwirkungen haben, also nicht direkt naschen! <3
	\item Jetzt nehmt Ihr Euch den hässlichsten von Omas Tellern und serviert Euch eine herrlich dampfende Suppe mit allem, was das Kellerherz begehrt: Fett, Gemüse, Elektrolyte und Fleisch.
}

\promisctext{Guten Appetit!}

\subsection{OC}
\prooc{474297-0}
