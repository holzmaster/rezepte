\prorezeptheader{
	Feuertopf
}{
	jessicahyde
}{
	428556
}{
	\diffwichtler
}

\hspace{10mm}

\promisctext{Dosen-Eintopf schmeckt nicht, aber Feuertopf ist geil. Also machen wir ihn nun selber.

\noindent Billiger, besser und in 'ner halben Stunde. Ohne viel Schneidearbeit!}

\hspace{10mm}

\promisctext{Ergibt: ~3-4 Liter Eintopf}

\hspace{10mm}

\prorezeptzutaten{
	\item 2 Dosen Kidneybohnen
	\item 1 große Dose Brechbohnen
	\item 1 kleine Dose Mais
	\item 1 Dose Pizzatomaten (ungewürzt)
	\item 2 Zwiebeln
	\item 2 Zehen Knoblauch
	\item 1 rote Paprika
	\item 2 Schoten Chili
	\item 1 Tck. Leberkäse grob
	\item Optional: 6 Miniwiener / 2 fr. Tomaten
	\item 500g gemischtes Hack
	\item 2 Pck. \glqq Fix für Chili con Carne\grqq
	\item Tomatenmark
	\item Cheyennepfeffer
	\item Schwarzer Pfeffer/Salz
}
\prorezeptequipment{
	\item Messer
	\item Herd
	\item Pfanne
	\item Kochtopf mit Deckel

	\propostqrinline{428556}
}

\prorezeptzubereitung{
	\item Zwiebeln schälen, klein schneiden, nicht hacken. Leberkäse würfeln.
	  Hack aus der Packung nehmen, kneten, mit Pfeffer/Salz würzen.
	\item Zwiebeln kurz anbraten, Fleisch dazugeben. Während alles brät, Paprika, Knoblauch und Chilis klein hacken, Dosen öffnen und überall das Einlegwasser abgießen.
	\item Wenn das Hack durch ist, zuerst das Fix für Chili hinzugeben, anschließend Knoblauch und Chilis.

	Umrühren, Wasser zugießen, bis Fleisch bedeckt. Dann alle Zutaten rein, zwei Esslöffel Tomatenmark einrühren, vorsichtshalber schon mal Pfeffer und Salz ja zwei Brisen drauf.
	\item Mit Deckel kochen lassen. Ungefähr eine halbe Stunde bis 45 Minuten, zwischendurch immer mal umrühren. Dann ein paar Minuten ziehen lassen, noch mal abschmecken und z. B. Zwiebelbaguette dazuessen.
}

\promisctext{Besser als der Doseneintopf? Siehste!}
