\prorezeptheader{
	Pizza
}{
	Duchess
}{
	474289
}{
	\diffneuschwuchtel
}

\hspace{10mm}

\prorezeptzutaten{
	\item \proimportant{Teig:}
		\item 500ml lauwarmes Wasser
		\item 925g glattes Mehl
		\item 42g frische Hefe
		\item 20g Salz
		\item 25ml Olivenöl
		\item 1 Prise Zucker
		\item KÄSE

	\item \proimportant{Soße:}
		\item 500m1 passierte Tomaten
		\item 1EL Tomatenmark
		\item Oregano
		\item Thymian
		\item Salz/Pfeffer
		\item 1TL Zucker
}
\prorezeptequipment{
	\item Große Schüssel
	\item Warmer Ort
	\item Nudelholz/Glasflasche
	\item Backofen

	\propostqrinline{474289}
}

\prorezeptzubereitung{
	\item Im lauwarmen Wasser löst ihr den ganzen Block (42g) frische Hefe auf. Danach gebt Ihr zu dem Wasser Olivenöl, Salz und Zucker hinzu.
	\item Das Mehl gebt Ihr in eine große Rührschüssel und schaufelt in der Mitte eine Kuhle. Langsam gebt Ihr immer ein bisschen von dem Wassergemisch hinzu und verknetet das Ganze zu einem glatten Teig. \proimportant{Ja, mit den Händen}.
	\item Der Teig muss klumpenfrei sein. Sobald Mehl und Wassergemisch vollständig miteinander vermengt sind, sollte der Teig auch nicht mehr klebrig sein. Wenn doch, vermengt noch einen Schluck Olivenöl und knetet den Teig noch einmal richtig durch.
	\item Der Teig muss nun an einem warmen Ort mindestens zwei Stunden ruhen. Also nicht unbedingt im Keller. Am besten Ihr stellt ihn auf's Fensterbrett über eine eingeschaltete Heizung. Achtet bitte darauf, dass der Teig durch die Hefe gehen wird und locker drei Mal an Größe zunimmt. Die Schüssel muss also entsprechend groß sein. Falls Ihr keine Schüssel habt, die groß genug ist, könnt Ihr den gekneteten Teig auch auf zwei Schüsseln aufteilen, um ihn ruhen zu lassen. Dieser Punkt ist wirklich wichtig, sonst ist die Pizza nachher hart wie Kruppstahl.
	\item Die Soße ist sehr simpel. Die passierten Tomaten (oder Pizzatomaten - jeder wie er mag) und Tomatenmark in einem Topf erhitzen. Dann mit den Gewürzen abschmecken. Neben Thymian und Oregano schmeckt auch Basilikum und Rosmarin sehr lecker. Dann salzen, pfeffern, den Zucker dazu geben und fertig.
	\item Der Teig ist nach zwei Stunden ausgeruht genug und wird nun noch einmal gut geknetet. Nun teilt Ihr den Teig mit einem Messer in sechs gleich große Stücke.
	\item Nun wird der Teig ausgerollt. Wenn Ihr kein Nudelholz habt, tut's auch eine Glasflasche. Rollt den Teig richtig dünn aus, sodass er gut auf ein Backblech passt. Als Unterlage könnte auch ein ausnahmsweise gewischter Fliesentisch fungieren.
	\item Den Backofen vorheizen auf volle Pulle. \proimportant{Ja, volle Pulle!} Der Ofen muss regelrecht glühen, dann wird die Pizza geil. Umluft ist am besten, aber auch ohne Umluftfunktion gelingt's.
	\item Den ausgerollten Teig bestreicht Ihr nun mit der Soße, die ein bisschen abgekühlt sein sollte. Zwei Esslöffel Sauce sind in etwa ausreichend.
	\item Und jetzt belegt Ihr dieses Prachtstück mit was auch immer Euch gefällt.
	\item Nun die Pizza mit Käse bestreuen. Ich empfehle Mozzarella, aber nehmt den, der Euch am besten schmeckt.
	\item Die Pizza muss bei voll vorgeheiztem Backofen \produration{7 bis 8 Minuten} in den Backofen. Dann ist sie goldbraun und genau richtig. Stellt Euch am besten einen Timer auf 6 Minuten und beobachtet dann die Pizza und nehmt sie raus, wenn sie euch gefällt. Das geht wirklich flott.

  	\noindent Auch ein guter Test: Wenn Ihr zwischendrin den Ofen kurz aufmacht und mit einer Gabel auf den Pizzarand drückt, sollte es knacken.
	\item Die fertige Pizza nach Belieben noch mit Oregano bestreuen und voilà!
	\item Der Teig hält sich problemlos zwei Tage (nicht in den Kühlschrank stellen!) daher sind die Angaben auch für 6 Pizzen: Morgens, mittags, abends. Morgens, mittags, abends.
}

\promisctext{Ich wünsche Euch gutes Gelingen und guten Appetit!}

\subsection{OC}
\prooc{474289-0}
